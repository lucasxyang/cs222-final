\section{Related Work}
\label{sec-related}
Compression is a well studied field outside of sensor networks such as
string based techniques LZ77~\cite{lz77} and LZW~\cite{lzw} and
entropy based techniques such as~\cite{cleary} and Huffman
coding~\cite{huffman}, but the unique challenges in performing
compressing on such resource constrained devices presents a new design
space and tradeoffs.

S-LZW~\cite{s-lzw} is a modification of LZW made to run on mote-class
devices.  They use dictionaries of sizes between 512 and 1024.  One
assumption they make is that the network is delay tolerant, which is
often not the case.  Our VBS delta compressing is done on the
streaming data, making is a good choice for networks that are not
delay tolerant.  

Barr and Asanovic~\cite{barr} looks at the power trade-off between
compressing and the transmitting and receiving of the data for PDA
class devices and 802.11 radios.  They used off the shelf compression
algorithms, which are too heavyweight to be run on mote class
devices.  Also, 802.11 radios have different power and data points
than that of an 802.15.4 radio.  

Strydis and Gaydadjiev~\cite{strydis} look at profiling a collection of
existing compression algorithms and a simulator for Intel XScale
cores, for implantable wireless biomedical devices.  As they are
targeting PDA/smartphone type processors, the compression algorithms
they evaluate are not designed to be run on mote class devices. 
