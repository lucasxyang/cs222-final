\section{LZ77}
\label{sec-lz77}

LZ77 \cite{lz77} seemed like a promising algorithm for Wireless Sensor
Network compression. Unlike LZW \cite{lzw} and other dictionary
building schemes, LZ77 maintains a constant size back-buffer with
which it performs matches. Without an upper bound on the dictionary
size, it is unreasonable to use dictionary based compression
algorithms because it is very easy to run out of memory on sensor
motes. LZ77, on the other hand, requires a constant amount of memory,
and using a pointer based circular buffer, it would be easy to
implement LZ77 on a mote without having to copy large pieces of
memory.

However, in practice, LZ77 does not perform as well. Further
investigation of LZ77 implementations revealed that these
implementations use large back-buffers, sometimes as big as 64k
entries. Many times, pure LZ77 is not implemented, but a history
dictionary is built in order to find matches more quickly. This
growing dictionary makes LZ77 prohibitive on sensor motes. Without a
history dictionary, matches can take a long time because the entire
back buffer must be searched for each new element that comes in. In
order to quantify how well LZ77 performs on our three data sets, we
compressed our data using a standard implementation of LZ77.

\begin{table}
  \begin{center}
  \begin{tabular}{|l||r|r|r|}
    \hline
    & LZ77 & RLE & Huffman \\ \hline
    Volcano & 99.7\% & 100.1\% & 81.6\% \\ \hline
    SHIMMER & 88.5\% - 98.8\% & 100.1\% & 70.8\% \\ \hline
    Marmot chirps & 98.5\% & 100.2\% & 70.2\% \\ \hline
    
  \end{tabular}
  \end{center}
  \caption{Compression ratios using standard methods}
  \label{table:compratios}
\end{table}

Table~\ref{table:compratios} shows the compression ratios for our
three data sets; high ratios mean bad compression. Note that LZ77
performs very poorly as the best compression ratio is 88.5\%. We used
a back-buffer size of 1024, which is a reasonable size for a mote
implementation. LZ77 performs so poorly for several reasons. First,
the back-buffer is not large enough to find a large amount of
matches. Also, the noise in the data makes matching very difficult. We
are not compressing english text, where there is no noise, instead we
are compressing sensor readings coming from an ADC, which have an
inherent amount of noise.

Table~\ref{table:compratios} also shows the compression ratios of two
other schemes: Run Length Encoding (RLE) and Huffman encoding. RLE
works by encoding consecutive duplicates with two numbers: the
element, and the length of the run. Due to the noise in our data,
there are very few runs, which is why RLE does so poorly. Huffman
coding does somewhat better, but it requires building a tree of prefix
codes. This tree could be very large, depending on how many bits each
sample has. For this reason, Huffman coding is not useful in a sensor
network context. 

These experiments have clearly shown that traditional techniques that
work in traditional computing environments, are not always suitable in
a sensor network context. We have not exhaustively tested all
compression schemes, but we have tested a few that are commonly
used. Moreover, complex algorithms are more difficult to implement on
sensor motes due to their limited debugging capabilities and lack of
mature development tools. We realize that the best compression scheme
for Wireless Sensor Networks is one that requires little computation,
is not very complex, and of course, has good compression ratios for
the sensor data.

\section{Variable Block Size Delta Encoding}
\label{sec-delta}

Given the inadequacies of standard compression schemes in the sensor
network context we developed a compression scheme that performs well
with sensor data, has minimal computation and is easy to implement on
motes. We decided to exploit the correlation between consecutive
samples. In sensor data, a sample is very likely to be close to the
previous sample. This characteristic lends itself nicely to delta
encoding. Because compression is very tied to data transmission in
sensor networks, our compression scheme must be designed for radio
transmission as well as the data we are compression. Whereas data
payload sizes in 802.11 packets are usually around 1500 bytes, data
payloads in 802.15.4 packets, which is the protocol used by sensor
motes, is between 30-100 bytes.

Given these requirements, we developed a variable block size delta
encoding scheme. For clarity of explanation, we will assume we are
working with 12 bit data and 30 byte packets. These characteristics
are those of the SHIMMER platform from which we obtained the
accelerometer data. However, this scheme works for higher order data
and larger packet sizes. All packets are fixed size, so we want to pack
as much information into a packet as possible. In addition, we want to
build some redundancy into the scheme. It might be possible to lose a
packet in sensor networks. However, since we are using delta encoding,
if we miss a packet, we still want to be able to decode the rest of
the data.

Each packet contains a block of data. Each block has a header and a
delta payload. The first sample in the block is stored in the header
uncompressed. This is for robustness so that if a packet is lost, only
the data in that packet is lost and not any subsequent data. The delta
is computed for every sample after the first with its previous
sample. The number of bits required to encode the delta is calculated
and stored. All of the deltas in the delta payload are encoded using
the same number of bits. The number of bits to encode the deltas is
therefore the number of bits needed to encode the largest delta. As
each sample comes in, the size of the block is calculated. If the
block still fits in the packet, the delta is added to the block. If
the block would not fit in the packet, the block is encoded into the
packet and dispatched. In the header we store the uncompressed first
element (12 bits), the block size (8 bits), and bits per delta (4
bits). The entire header fits into 3 bytes, leaving 27 bytes for the
delta payload. There are two reasons why a new sample would not fit in
the block. First, if the block already has many samples, a new delta
might simply not fit in the block. Secondly, if a very large delta
comes in, it would require that all the other deltas be encoded with
many more bits, which would push the entire block over the packet size
limit. 

This compression scheme is extremely simple to implement and requires
very little computation. For each sample, a delta must be calculated,
and it must be compared with the largest delta. When transmitting the
block, the deltas must be packed into the payload. Compared to LZ77 or
Huffman coding, this is very little computation. In the following
section we will evaluate this compression scheme using our data sets
and determine how well it performs. We will also quantify the effect
of noise on our scheme.

