\section{Introduction} 

Wireless sensor networks involve collections of many battery operated
nodes that can number in the thousands.  A canonical example of a
sensor node is the Telos Mote~\cite{telos} (colloquially referred to
as a mote) which has a 16-bit microcontroller running up to 8 MHz, 48
KB of program memory, 10 KB of ram, and a 2.4 GHz 802.15.4 radio with
an effective data rate of 250 kbps.  This is five orders of magnitude
differnce in memory size, four orders of magnitude of difference is
networking speed, and three orders of magnitude difference in
processing speed than a typical desktop computer or laptop.  These
sensor networks are often powered by only a pair AA batteries and are
expected to last months or even years on such a limited power source.

As implied in their name, the primary goal of a sensor networks is to
sense and report on their surrounding habitat. As the field has
matured from some of the early low data-rate habitat monitoring
applications such as Great Duck Island~\cite{gdi-sensys04}, a new
class of sensor network applications have emerged requiring high data
rates, high data fidelity requirements, and computationally intensive
processing. Examples of applications in this class include vibration
monitoring of bridges and
buildings~\cite{brimon,netshm-spots06,ggb-monitoring}; seismic
monitoring of fault zones and
volcanoes~\cite{ucla-seismic,volcano-osdi06}; acoustic monitoring of
animal habitats~\cite{girod-marmots,enviromic}; and body sensor
networks for monitoring activity and
movement~\cite{intel-msp,satire,parkinsons-embs07}.  These
applications present several unique challenges to sensor networks.
The amount of data being generated by these systems outstrips the
radio's ability to transmit this data (say to a base station).
Compression is one way to help cope with the volume of data that these
applications produce.  

Due to limited energy available to these systems, reducing power is of
primary importance.  Of the energy costs, radio transmissions and
receptions dominate the energy budget.  The radio, both in
transmission and reception, uses an order of magnitude more power than
just processing alone~\cite{telos}.  Because of the high energy costs
to send and receive data, there can be a large potential energy
savings if the sensor nodes are able to compress their data before
transmitting.

Because most sensor networks are spatially many times larger than
their radio range, sensor networks often use multi-hop routing move
data across the network.  Because of multi-hop routing, compression
not only obtains a savings of energy at the node level, but also at
any node that this data passes through on its way to its destination. 

But most sensor networks do not use data compression.  The problem is
that because of the limited processing power and the extremely
constrained memory on these devices, implementing traditional
compressions schemes are non-trivial.  For example, many compression
schemes, like LZ77~\cite{lz77}, use large data buffers in the
compression algorithm.  Sensor network nodes, with their extremely
limited memory budget will not be able to have such a large window,
and compression ratios will suffer.

\subsection{Data Sources} 

The type of data being compressed plays an important factor in how
well a compression algorithm will do.  Certain types of data will
favor one compression scheme over another compression scheme.  We
evaluate several compression schemes on three types of sensor data
collected from actual sensor network deployments.  

\subsubsection{Seismic Volcano Data}

