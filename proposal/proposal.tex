\documentclass[10pt]{article}
\usepackage{graphicx, float}
\usepackage{tabularx}
\usepackage{fullpage}
\pdfpagewidth 8.5in
\pdfpageheight 11in
\begin{document}
\begin{center}
\Large\textbf{CS222 Project Proposal \\ Data Compression on Wireless Sensor Networks}

\bigskip

Jason Waterman \\ Daniel Shteremberg \\
\medskip
March 31st 2009
\end{center}

\section{Introduction}

Wireless Sensor Networks present a unique computing environment due to
their limited resources. Many traditionally accepted paradigms must be
reexamined in this new context. One important contraint is energy,
which often requires reengineering of old techniques to meet the new
energy requirements. Compresseion is an example of such a
technique. Most compression schemes were designed with performance and
compression ratios as the primary metrics of success. Rarely was
energy efficiency a metric in the design of such schemes. In Wireless
Sensor Networks, energy efficiency is vital and any compression
implementation must be tested for energy efficiency as well as
performance and compression ratios. 

We propose to implement several compression schemes on a wireless
sensor mote. We will test the energy efficiency, as well as
compression and performance, directly on the mote. These experiments
will tell us how feasable it would be to use traditional compression
schemes in Wireless Sensor Networks. We will determine which schemes
are the most energy efficient, or if new compression schemes must be
designed with energy efficiency in mind. Due to sensor motes' limied
memory resources, we will also determine if it is possible to
implement these compression schemes on such small memory devices. 

\section{Related Work}

Many papers in the Wireless Sensor Network field deal with compression
on the side rather than as the primary focus
\cite{habitat}\cite{structural}. Xu et al. \cite{structural} mention
compression and briefly describe a wavelet compression scheme, but
never implement it on a mote. These scheme transmits wavelet
compressed (lossy) data, but stores the raw uncompressed data in
flash. They also propose quantization and thresholding as a simple way
to achieve lossy compression. These techniques might be well suited
for wireless sensor networks, but this is not the primary focus of
their paper.

Mainwaring et al. \cite{habitat} also mention compression in their
paper as a side note. They state that compression is a trade off
between the cost of data processing and the cost of data
transmission. We believe this is the right approach when examining
compression schemes on motes. They present anecdotal evidence that was
obtained using compression schemes run on Unix. They compare various
resolutions of data, delta encoding, Huffman encoding, gzip, and
bzip2. Although these results are a good first step, they never
implement any of these algorithms on a mote and do not measure the
energy cost of these algorithms. 

Barr et al. \cite{energyaware} perform a thorough energy analysis of
several compression schemes. This analysis would be applicable to
Wireless Wensor Networks if they performed their experiments on mote
class devices. Instead they use a StrongARM/Linux based platform with
an 802.11 wireless interface. We intend to mimic the type of analyis
that these authors performed, on a mote class device, with a TinyOS
implementation of the compression schemes studied. In addition, these
authors used data from the Calgary Corpus. This corpus contains many
english texts and is not representative of the type of data
encountered in Wireless Sensor Networks. We will use a more
appropriate corpus for our experiments.

\section{Proposed Approach} 

First we will implement the LZW \cite{lzw} compression algorithm in
TinyOS. This type of sliding window algorithm is well suited for the
low memory constraints of a sensor mote. Implementing this algorithm
on such a small memory will be an important challenge and one of the
goals of this project.

Using sensor data obtained through previous experiments, we will run
experiments to determine the cost of compressing away one bit of
information. Due to the multi-hop nature of many wireless sensor
networks, we believe that the cost of compressing one bit will be
amortized over many hops. The one bit compression cost will help
determine the optimal routing tree depth in terms of reducing
compression cost. 

Once we have finished implementing LZW we will investigate if other
popular compression schemes are suitable for implementation on
motes. If so, we will implement them as we did LZW and perform the
same type of energy analysis. 

\section{Resources}

We will need at least two tMote Sky devices and the TinyOS
toolchain. We have these devices and tools available and the
experience necessary to work with this platform. We might also want to
run experiments on MoteLab, which is a 184 mote platform available to
us. We will also need a tool to measure the energy expended by the
motes, which we also have available. 

\bibliographystyle{plain}
\bibliography{proposal}
\end{document}
