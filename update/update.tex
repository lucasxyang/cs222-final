\documentclass[10pt]{article}
\usepackage{graphicx, float}
\usepackage{tabularx}
\usepackage{fullpage}
\pdfpagewidth 8.5in
\pdfpageheight 11in
\begin{document}
\begin{center}
\Large\textbf{CS222 Project Update \\ Data Compression on Wireless Sensor Networks}

\bigskip

Jason Waterman \\ Daniel Shteremberg \\
\medskip
April 30th 2009
\end{center}

\section{Recap}

The goal of this project is to analyze the performance and energy
costs of compression in Wireless Sensor Networks. We proposed to
implement the LZ77 compression algorithm on a mote and measure the energy
associated with compressing one bit of information. We also proposed
to examine three data sets obtained from real life Wireless Sensor
Network deployments and determine their compressibility. Finally,
armed with compression ratios for real life data, and energy costs
associated with compression, we proposed to examine the trade-offs
related to compression. We want to answer the question of \"Will
compressing data result in energy savings when the data will be
transmitted?\" In addition, we proposed to consider multi-hop routing
topologies. If it is more expensive (energy wise) to compress the
data than to send the uncompressed data, we proposed to examine how
deep the routing trees must be in order to amortize the energy savings
over the entire network as the compressed packet moves up the tree. 

\section{Accomplished Tasks}

We have obtained three data sets from real-life Wireless Sensor
Network deployments. The first is a set of accelerometer and gyro data
from a body sensor network. The second is the acoustic data of marmot
sounds. The third is the seismic data of a network deployed to monitor
a volcano. We have calculated the entropy of some of these data sets
and found that so far they are not very compressible. However, it is
theoretically possible to compress this data by 25\%. 

We have an LZ77 implementation on a mote. We are currently in the
process of loading the data sets into the flash memory of the mote
devices and running the compression algorithms on the motes. 

We have implemented a simulator that will help us examine the
trade-offs once we have the energy data. This simulator will make this
analysis easier because we will no longer have to be running
experiments on the motes, which is often a tedious and time consuming
task. 

\section{Future Tasks}

We plan to measure the energy associated with compressing this data
directly from the mote. We will remove the overhead of reading and
writing to flash by performing the experiment without
compression. Then we will run the experiment with compression and take
the difference to remove the overhead.

Once we have the energy data, we will use our simulator to examine the
trade-offs related to compression. Our simulator will also help us
examine multiple routing topologies and determine if compression is
worth the radio power savings under these circumstances. One common
problem in Wireless Sensor Networks is that a node is not usually able
to transmit power to a neighbor if the neighbor requires it. However,
with compression we can do something similar. If a node has data to
transmit, it can compress the data (even if it will waste more
energy), but the next node in the routing tree will save energy
because the data is not as large. In essence, a node is able to pass
on the energy savings of compression to another node and this is a
useful mechansim that we want to explore in our simulations. 

We have about 2 weeks left to finish our project, which we believe is
sufficient time to accomplish our future tasks and write the paper.


%\bibliographystyle{plain}
%\bibliography{update}
\end{document}
